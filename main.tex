\documentclass[oneside]{tongjithesis}
\usepackage{tongjithesis}

%%%%%%%%%%%%%%%%%%%%%%%%%%%%%%%%%%%
% 若你想要用句点(.)替换句号(。)
% 可以打开下面的注释
%%%%%%%%%%%%%%%%%%%%%%%%%%%%%%%%%%%
% \catcode`\。=\active
% \newcommand{。}{\ifmmode\text{.}\else .\fi}
%%%%%%%%%%%%%%%%%%%%%%%%%%%%%%%%%%%

%%%%%%%%%%%%%%%%%%%%%%%%%%%%%%%%%%%
% 在Windows上使用时:
% 若你想避免将 Python 加入环境变量,可以在此手动指定python环境
% 打开下面的注释,并替换尖括号<>中的路径
% e.g. {<X:/your/path/to/python>/python.exe} -> {C:/software/Anaconda3/envs/latex/python.exe}
%%%%%%%%%%%%%%%%%%%%%%%%%%%%%%%%%%%
% \newcommand{\myPython}{<X:/your/path/to/python>/python.exe}
% \renewcommand{\MintedPygmentize}{\myPython\space -m pygments}
%%%%%%%%%%%%%%%%%%%%%%%%%%%%%%%%%%%

\begin{document}

\input{sections/frontcover.tex}

\makeatletter
\@ifclasswith{tongjithesis}{twoside}{%
\cleardoublepage
}{%
\clearpage
}
\makeatother

\pagestyle{firststyle}
\MakeAbstract{
    摘要通常是一篇文章、论文、报告或其他文本的简短概括。它的目的是帮助读者了解文本的主要内容和结论,以便决定是否需要继续阅读原文。摘要通常包含文本的主题、目的、方法、结果和结论等方面的信息,并尽可能简洁明了地呈现。好的摘要应该能够概括文本的重点,同时避免使用不必要的细节和专业术语,以便广大读者能够轻松理解。

    此外,摘要通常也是学术界和研究人员评估一篇文献的重要依据之一。在文献检索和筛选过程中,人们通常会根据摘要来决定是否进一步查看完整的文献。因此,撰写一个清晰、准确、简洁的摘要对于文献的传播和影响力至关重要。在撰写摘要时,作者应该遵循文献的格式要求和撰写规范,同时结合文本的内容和目的,将摘要撰写得准确、简洁、易懂,以提高文献的传播和影响力。

    关键词1,关键词2,关键词3通常是与文章内容相关的几个词语,用于帮助读者更好地了解文章主题和内容。关键词的选择应该与文章的主题和研究领域密切相关,通常应该选择具有代表性、权威性、独特性和可搜索性的词语。
}{关键词1,关键词2,关键词3}

\makeatletter
\@ifclasswith{tongjithesis}{twoside}{%
  \cleardoublepage
}{%
}
\makeatother

\MakeAbstractEng{
    An abstract is usually a short summary of an article, essay, report, or other text. Its purpose is to help the reader understand the main content and conclusions of the text so that he or she can decide whether he or she needs to continue reading the original text. The abstract usually contains information about the topic, purpose, methods, results, and conclusions of the text and is presented as concisely and clearly as possible. A good abstract should be able to summarize the main points of the text while avoiding unnecessary details and jargon so that it can be easily understood by a wide audience.

    In addition, abstracts are often one of the most important bases on which academics and researchers evaluate a piece of literature. During the literature search and selection process, people often base their decision to look further into the complete literature on the abstract. Therefore, writing a clear, accurate, and concise abstract is crucial to the dissemination and impact of the literature. When writing an abstract, authors should follow the formatting requirements and writing specifications of the literature, as well as combine the content and purpose of the text to write an accurate, concise, and easy-to-understand abstract in order to improve the dissemination and impact of the literature.

    Keyword 1, Keyword 2, and Keyword 3 are usually a few words related to the content of the article and are used to help readers better understand the topic and content of the article. The choice of keywords should be closely related to the topic and research area of the article, and words that are representative, authoritative, unique, and searchable should usually be chosen.
}{Keyword 1, Keyword 2, Keyword 3}


\makeatletter
\@ifclasswith{tongjithesis}{twoside}{%
\cleardoublepage
}{%
\clearpage
}
\makeatother

\tableofcontents   %放置目录

\makeatletter
\@ifclasswith{tongjithesis}{twoside}{%
\cleardoublepage
}{%
\clearpage
}
\makeatother

\pagestyle{mainstyle}
\section{介绍}\label{sec:introduction}

在本节(\cref{sec:introduction})中,我们将讨论文档中常用的各种标题级别和字体样式。在文章中,标题是非常重要的组成部分之一,它可以帮助读者更好地理解文章的结构和内容。

\subsection{二级标题}

二级标题是一种较为重要的标题级别,一般用于表示文章中的主要章节或主题。通常,它们会在上面添加分割线或加粗等效果,以突出其重要性。

\subsubsection{三级标题}

相对于二级标题而言,三级标题是更加具体的标题级别,通常用于表示二级标题下的具体内容描述。它们的长度通常比二级标题短,与二级标题之间应有一定的间距。

\paragraph{段落标题}

段落标题是文章中比正文稍微具有一些重要性和突出性的内容,通常用加粗或斜体等方式来区别于正文。

\subparagraph{子段落标题}

子段落标题是相对于段落标题更加细节化的内容,用于突出一段文字中的重点内容。通常采用斜体或加粗的方式表示。在一些正式的文献中,子段落标题的使用较少。

\subsection{字体}

在下面的段落中,我们使用了不同的字体来表示不同的文字信息。下面是各段落所使用的字体和对应的命令:

\begin{itemize}
    \item {\songti 宋体}:使用命令 \texttt{\textbackslash songti}。
    \item {\heiti 黑体}:使用命令 \texttt{\textbackslash heiti}。
    \item {\fangsong 仿宋}:使用命令 \texttt{\textbackslash fangsong}。
    \item {\kaishu 楷书}:使用命令 \texttt{\textbackslash kaishu}。
\end{itemize}

{\songti

1900年前后,由埃里希·宝隆创办的“同济医院”正式挂牌。医院的医师大都是“德医公会”成员。他们白天忙于经营自己的诊所,只有傍晚到医院看门诊、动手术。埃里希·宝隆医生看到医院里的医疗力量不足,计划在院内设立一所德文医学堂,招收中国学生,以培养施诊医生。这个计划得到德国驻沪总领事以及德国政府高等教育司的支持。1906年,他们设立了一个支持医学堂开办的基金会,得到了德国“促进德国与外国思想交流的科佩尔基金会”的协助,筹集到一批医科书刊及新式的外科手术电动器械等物品。}

{\heiti 1907年6月医学堂开学前,德国驻沪总领事克纳佩在上海不仅号召德国商人捐款,而且要求德国洋行向中国商人募捐。同时,费舍尔还要求中国官方的资助和支持,克纳佩利用在中德两国募来的捐款,成立了“为中国人办的德国医学堂基金会”。当时规定,捐款金额较多者可成为医学堂董事会董事。医学堂建立时定名为德文医学堂,并成立了董事会负责学校的管理。董事会由18人组成,主要成员有:三个德医公会元老:宝隆、福沙伯(第二任校长)、福尔克尔;三名德国商人:莱姆克、米歇劳和赖纳;两名中国绅商:朱葆三(沪军都督府财政部长及上海商务会会长,大买办)、虞洽卿(荷兰银行买办);总领事馆的副领事弗赖海尔·冯·吕特等。埃里希·宝隆医生被正式推选为董事会总监督(董事长)兼学堂首任总理(校长),负责学堂的管理。医学堂的校址设在同济医院对面的白克路(今凤阳路415号上海长征医院内)。1907年10月1日德文医学堂举行了开学典礼。}

{\ifcsname fangsong\endcsname\fangsong\else[无 \cs{fangsong} 字体。]\fi
1923年3月17日北洋政府教育部下达第108号训令,批准同济工科“改为大学”。学校随即召开董事会议,将学校定名为“同济大学”。1923年3月26日,学校以“同济大学董事会”名义呈文北洋政府教育部,称“经校董会议定名称为同济大学”。1923年4月24日,北洋政府教育部下达第634号“指令”,称“该校名称拟改为同济大学,应予照准备案”。1924年5月20日北洋政府教育部下达第120号训令,批准同济医科为大学。从此以后,5月20日定为校庆日。}

{\ifcsname kaishu\endcsname\kaishu\else[无 \cs{kaishu} 字体。]\fi
抗日战争胜利后,1946年,国立同济大学分批迁回上海。由于缺少校舍,学校分散教学,成为斜跨上海市区、分散十多处的“大学校”。其中学校办公室和医学院位于善钟路100弄10号(今常熟路),附属医院分别为白克路上的中美医院(原宝隆医院,今凤阳路)和同孚路82号(今石门一路)的原德国医院,理学院在平昌街日本第七国民学校内(今国顺路上海电视大学),文法学院在四川北路(今复兴初级中学),新生院位于江湾新市区的市图书馆(今黑山路),高级工业职业学校则在江湾魏德迈路(今邯郸路),附属中学位于市博物馆(今长海医院飞机楼)。工学院位于其美路的原日本中学(今杨浦区四平路1239号),1949年后逐步发展成同济大学的主校区。}

\subsubsection{测试生僻字}\label{sec:uncommon}

本小节对生僻字\footnote{此处的生僻字指:GBK编码中有,但GB2312编码中没有的字。}的显示进行测试。

丂丄丅丆丏丒丗丟丠両丣並丩丮丯丱丳丵丷丼乀乁乂乄乆乊乑乕乗乚乛乢乣乤乥乧乨乪乫乬乭乮乯乲乴乵乶乷乸乹乺乻乼乽乿亀亁亃亄亅亇亊亐亖亗亙亜亝亣亪亯亰亱亴亶亷亸亹亼亽亾仈仌仏仐仒仚仛仜仠仢仦仧仩仭仮仯仱仴仸仹仺仼仾伀伂伃伄伅伆伇伈伋伌伒伓伔伕伖伜伝伡伣伨伩伬伭伮伱伳伵伷伹伻伾伿佀佁佂佄佅佇佈佉佊佋佌佒佔佖佡佢佦佨佪佫佭佮佱佲併佷佸佹佺佽侀侁侂侅侇侊侌侎侐侒侓侕侘侙侚侜侞侟価侢侤侫侭侰侱侲侳侴侶侷侸侹侺侻侼侽侾俀俁係俆俇俈俉俋俌俍俒俓俔俕俖俙俛俢俤俥俧俫俬俰俲俴俵俶俷俹俻俼俽俿倀倁倂倃倄倅倇倈倊倎倐倓倕倖倗倛倝倞倠倢倣値倧倯倰倱倲倳倴倵倶倷倸倹倻倽倿偀偁偂偄偅偆偊偋偍偐偑偒偓偔偖偗偘偙偛偝偞偟偠偡偢偣偤偦偧偨偩偪偫偭偮偯偰偱偲偳偸偹偺偼偽傁傂傃傄傆傇傉傊傋傌傎傏傐傑傒傓傔傕傖傗傚傛傜傝傞傟傠傡傢傤傦傪傫傮傯傰傱傴傶傸傹傼傽傿僀僁僂僃僄僆僇僈僉僊僋僌働僎僐僒僓僔僗僘僙僛僜僝僟僠僡僢僣僤僨僩僪僫僯僰僱僲僴僶僷僸僺僼僽僾僿儁儂儃儅儊儌儍儎儏儐儑儓儔儕儖儗儘儙儚儛儜儝儞儠儢儣儤儥儦儧儨儩儫儬儭儮儯儰儱儳儴儵儶儷儸儹儺儻儼儽。
\clearpage
\section{数学}\label{sec:math}

在本节(\cref{sec:math})中,我们展示各种数学符号和环境的使用。

\subsection{数字和单位}

\subsubsection*{数字格式化}

数字可以使用 \verb|\num{}| 命令进行格式化,例如:

\begin{itemize}
  \item \num{12345272.42146}:显示长数字。
  \item \num{9+-3i}:显示复数。
        % \item \complexnum{1+-2i}  % For TeXLive 2021, siunitx >= 3.0
  \item \num{9.624 x 8.18 x 1.745}:显示乘积表达式。
        % \item \numproduct{1.654 x 2.34 x 3.430}  % For TeXLive 2021, siunitx >= 3.0
  \item \num{1.234567e8}:以科学计数法显示数字。
  \item \num{-19.7}:显示负号。
  \item \num{3.2d-5}:显示数字和指数,使用小写的 d 来指示指数。
  \item \num{1.234(5)}:显示带有不确定度的数字。数字后的括号内的数字表示数字的不确定度,本例中为 5。
  \item \num[parse-numbers=false]{\$19,456.82}:显示一个货币金额,并将其解释为一个字符串,而不是数字。
\end{itemize}

\subsubsection*{单位格式化}

单位可以使用 \verb|\si{}| 命令进行格式化,例如:

\begin{itemize}
  \item \si{kg.m.s^{-1}}:显示物理量的单位。
  \item \si{\pico\meter} $\si{\pico\meter}$:显示物理量的单位,可以添加前缀。
  \item \si{\tesla} $\si{\tesla}$:显示物理量的单位。
  \item \si{\ampere\per\meter\squared}:显示物理量的单位,并使用 \verb|\per| 命令来显示单位的分数形式。
  \item \si[per-mode=fraction]{\coulomb\per\mole}:显示物理量的单位,并使用分数形式显示单位,而不是使用倒数形式。
  \item \si{\kilo\hertz}:显示带有前缀的单位。
  \item \SI{2.5}{\kilo\volt\per\milli\meter}:显示物理量的值和单位,其中值为 2.5,单位为 \si{\kilo\volt\per\milli\meter}。
  \item \SI{12}{\degree}:显示一个角度值,单位为度。
\end{itemize}

\subsubsection*{数字和单位组合}

数字和单位可以组合起来使用,例如:

\begin{itemize}
  \item \SI{0.5}{\metre\cubed}:显示物理量的值和单位。
  \item \SI{5.5}{\micro\second}:显示物理量的值和单位。
  \item \SI{2.3}{\mega\joule\per\kelvin}:显示物理量的值和单位。
\end{itemize}

\subsubsection*{数字和单位列表}

数字和单位列表可以使用以下命令进行格式化:

\begin{itemize}
  \item \numlist{95;36}:显示数字列表,其中数字之间用分号隔开。
  \item \numlist{52;74;59}:显示数字列表,其中数字之间用分号隔开。
  \item \SIlist{0.53;0.74;0.99}{\milli\metre}:显示带有单位的数字列表,其中数字之间用分号隔开。
\end{itemize}

\subsubsection*{数字和单位范围}

数字和单位范围可以使用以下命令进行格式化:

\begin{itemize}
  \item \numrange{65}{1284}:显示数字范围,其中数字用短横线隔开。
  \item \SIrange{58}{1024}{\degreeCelsius}:显示带有单位的数字范围。
\end{itemize}

以上是一些常见的使用示例,宏包 \pkg{siunitx} 还提供了许多其他选项和命令,可以根据需要进行自定义设置。

\subsection{数学符号和公式}

按照国标 GB/T 3102.11—1993《物理科学和技术中使用的数学符号》,
微分符号 $\dd$ 应使用直立体。除此之外,数学常数也应使用直立体:
\begin{itemize}
  \item 微分符号 $\dd$:\cs{dd}
  \item 圆周率 $\uppi$:\cs{uppi}
  \item 自然对数的底 $\ee$:\cs{ee}
  \item 虚数单位 $\ii$, $\jj$:\cs{ii} \cs{jj}
\end{itemize}

公式应另起一行居中排版。公式后应注明编号,按章顺序编排,编号右端对齐。
\begin{equation}
  \frac{\dd^2}{\dd t^2}x(t)+\omega^2x(t)=\sin(t)
\end{equation}
\begin{equation}
  \frac{\partial^2u}{\partial t^2}=\frac{\partial^2u}{\partial x^2}+\frac{\partial^2u}{\partial y^2}
\end{equation}

公式末尾是需要添加标点符号的,至于用逗号还是句号,取决于公式下面一句是接着公式说的,还是另起一句。
\begin{equation}
  \sum_{n=1}^{\infty}\frac{\cos(nx)}{n^2}=\frac{\uppi^2}{6}-\frac{x^2}{4}
\end{equation}

公式较长时最好在等号“$=$”处转行。
\begin{align}
       & I(X_1;X_2)-I(X_1;X_2|X_3) \nonumber    \\
  =    & H(X_2)-H(X_2|X_3) \nonumber            \\
  =    & H(X_2,X_3)-H(X_3)-H(X_2|X_3) \nonumber \\
  =    & I(X_2;X_3)-I(X_2;X_3|X_1) \nonumber    \\
  =    & I(X_2;X_3,X_1) \nonumber               \\
  \geq & 0
\end{align}

如果在等号处转行难以实现,也可在 $+$、$-$、$\times$、$\div$ 运算符号处转行,转行时运算符号仅书写于转行式前,不重复书写。
\begin{multline}
  \frac{1}{2} \Delta (f_{ij} f^{ij}) =
  2 \left(\sum_{i<j} \chi_{ij}(\sigma_{i} - \sigma_{j})^{2}
  + f^{ij} \nabla_{j} \nabla_{i} (\Delta f)  + \nabla_{k} f_{ij} \nabla^{k} f^{ij} +
  f^{ij} f^{k} \left[2\nabla_{i}R_{jk}- \nabla_{k} R_{ij} \right] \vphantom{\sum_{i<j}} \right) \\
  - 3H^{2}\left[1+\frac{\dot{\phi}}{2H^2}\right] - \frac{\dot{\phi}^{2}}{2} - \frac{k}{a^{2}}\phi^{2} - \frac{1}{2}\left(\frac{\partial\phi}{\partial t}\right)^{2} + \frac{a^2}{2}\left(\frac{\partial\phi}{\partial x}\right)^{2}
  + \frac{1}{4}\lambda\phi^{4} + \frac{\beta}{3}\phi^{3} \\
  -\frac{1}{2}\mu^{2}\phi^{2}(\ln{\phi^{2}} - c)
  + \frac{e^{2}}{2}\left(\frac{\partial A_{\mu}}{\partial t}\right)^{2} - \frac{e^{2}}{2}\left(\frac{\partial A_{\mu}}{\partial x}\right)^{2}
  - e^{2}\phi^{2}A_{\mu}A^{\mu} + \frac{1}{4}F_{\mu\nu}F^{\mu\nu}.
\end{multline}

\subsection{定理环境}

定理环境是数学领域中常用的一种环境类型,用于表述和证明各种数学定理、推论、引理等。在 \LaTeX\ 中,我们可以通过定义定理环境来方便地排版定理等内容,使其在文档中具有更加统一的风格和格式。

在本节中,我们展示了一些常用的定理环境,包括定理、推论、引理、命题、猜想、假设、定义、例、注等。这些环境的具体样式可以通过修改定理环境的定义进行自定义。此外,我们还展示了定理证明环境,用于证明各种定理和结论。在定理证明环境中,我们可以使用各种数学符号和公式,方便地展示证明过程,使读者更加易于理解。

请注意,下面的文字均由ChatGPT生成。我们不保证它们的正确性。

\begin{thm}
  对于任意正整数 $n$,存在无穷多个素数。
\end{thm}

\begin{proof}
  我们采用反证法。假设存在一个最大的素数 $p$,那么任何大于 $p$ 的正整数 $n$ 都可以表示为 $p$ 和某个正整数 $k$ 的积,即 $n = pk$。但是根据这个假设,$p$ 已经是最大的素数了,所以 $k$ 不可能再包含其他的素数。这意味着我们可以将所有大于 $p$ 的正整数分成两个不相交的集合,一个是由 $p$ 的倍数组成的集合,另一个是由剩下的数组成的集合。而根据素数的定义,这两个集合中至少有一个是不存在素数的。然而,由于 $p$ 是最大的素数,所以它肯定不在由 $p$ 的倍数组成的集合中。这就产生了矛盾,因此假设不成立,定理得证。
\end{proof}

\begin{cor}
  如果 $a$ 和 $b$ 是正实数,则 $a+b \geq 2\sqrt{ab}$。
\end{cor}

\begin{proof}
  我们可以使用平均数不小于几何平均数这个不等式来证明。具体地,我们有:
  \begin{align*}
    \sqrt{ab}              & \leq \frac{a+b}{2} \\
    \Rightarrow 2\sqrt{ab} & \leq a+b
  \end{align*}
  因此,$a+b \geq 2\sqrt{ab}$,定理成立。
\end{proof}

\begin{lem}
  设 $f$ 是一个在 $[a,b]$ 上的可微函数,则 $f$ 在 $[a,b]$ 上一定有一点使得 $f'(x) = \frac{f(b)-f(a)}{b-a}$。
\end{lem}

\begin{proof}
  我们定义一个新的函数 $g(x) = f(x) - \frac{f(b)-f(a)}{b-a} \cdot (x-a)$。这个函数满足 $g(a) = f(a)$,$g(b) = f(b)$,且 $g'(x) = f'(x) - \frac{f(b)-f(a)}{b-a}$。如果我们能够找到一个点 $c \in (a,b)$,使得 $g(c) = 0$,那么根据罗尔定理,$g'(x)$ 必然在 $[a,b]$ 中的某个点为 $0$,即存在 $x_0 \in (a,b)$,使得 $f'(x_0) = \frac{f(b)-f(a)}{b-a}$。因此,我们只需要证明 $g(a)$ 和 $g(b)$ 异号即可。由于 $f$ 在 $[a,b]$ 上连续,根据介值定理,存在一个点 $c \in [a,b]$,使得 $f(c) = \frac{f(b)+f(a)}{2}$。那么,我们有:
  \begin{align*}
    g(a) & = f(a) - \frac{f(b)-f(a)}{b-a} \cdot (a-a) \\
         & = f(a)                                     \\
         & < \frac{f(b)+f(a)}{2}                      \\
         & = f(c)                                     \\
         & > \frac{f(b)-f(a)}{b-a} \cdot (c-a)        \\
         & = f(a) - \frac{f(b)-f(a)}{b-a} \cdot (b-a) \\
         & = g(b)
  \end{align*}
  因此,$g(a)$ 和 $g(b)$ 异号,定理得证。
\end{proof}

\begin{prop}
  设 $f$ 和 $g$ 是两个单调递增的函数,则 $f+g$ 也是一个单调递增函数。
\end{prop}

\begin{proof}
  我们需要证明对于任意的 $x,y \in \mathbb{R}$,如果 $x < y$,则 $f(x)+g(x) < f(y)+g(y)$。由于 $f$ 和 $g$ 都是单调递增的,所以有 $f(x) < f(y)$ 和 $g(x) < g(y)$,于是 $f(x)+g(x) < f(y)+g(x) < f(y)+g(y)$。因此,$f+g$ 是单调递增的,定理得证。
\end{proof}

\begin{conj}
  存在无穷多个双素数,即形如 $p(p+2)$ 的素数。
\end{conj}

\begin{proof}
  我们采用反证法。假设存在一个最大的双素数 $p(p+2)$,那么任何大于 $p(p+2)$ 的正整数 $n$ 都可以表示为 $p(p+2)$ 和某个正整数 $k$ 的积,即 $n = p(p+2)k$。但是根据这个假设,$p(p+2)$ 已经是最大的双素数了,所以 $k$ 不可能再包含其他的双素数。这意味着我们可以将所有大于 $p(p+2)$ 的正整数分成两个不相交的集合,一个是由 $p(p+2)$ 的倍数组成的集合,另一个是由剩下的数组成的集合。而根据双素数的定义,这两个集合中至少有一个是不存在双素数的。然而,由于 $p(p+2)$ 是最大的双素数,所以它肯定不在由 $p(p+2)$ 的倍数组成的集合中。这就产生了矛盾,因此假设不成立,定理得证。
\end{proof}

\begin{assume}
  假设 $A$ 和 $B$ 是两个集合,则它们的笛卡尔积 $A \times B$ 的基数为 $|A| \cdot |B|$。
\end{assume}

\begin{proof}
  我们需要证明 $|A \times B| = |A| \cdot |B|$。我们可以使用等势关系来证明,即构造一个双射 $f: A \times B \rightarrow \left\{1,2,\dots,|A| \cdot |B|\right\}$。具体地,我们可以按照以下方式定义 $f$:
  \begin{equation}
    f(a,b)=(a-1)\cdot |B|+b
  \end{equation}
  其中 $a \in A$,$b \in B$。显然,$f$ 是一个双射,因为每个元素 $(a,b)$ 都对应着唯一的自然数 $(a-1) \cdot |B| + b$,而每个自然数也都可以唯一地表示为 $(a-1) \cdot |B| + b$ 的形式。因此,$A \times B$ 和 $\left\{1,2,\dots,|A| \cdot |B|\right\}$ 等势,即它们有相同的基数,定理得证。
\end{proof}

\begin{dfn}
  一个有向无环图是拓扑排序的,当且仅当它的所有顶点都可以按照某个顺序进行编号,使得对于任意一条有向边 $(u,v)$,$u$ 的编号都小于 $v$ 的编号。
\end{dfn}

\begin{exmp}
  考虑一个任务调度问题,其中有 $n$ 个任务需要按照一定的顺序执行。如果任务之间存在依赖关系,即某些任务必须在其他任务执行完毕之后才能开始执行,那么我们可以将这些任务和它们之间的依赖关系表示成一个有向无环图,并通过拓扑排序来确定任务的执行顺序。
\end{exmp}

\begin{rem}
  在数学中,定义和定理是数学推理的基础,而引理和命题则通常作为定理证明的中间步骤。而且在数学研究中,猜想和猜测往往需要大量的证明和反证,才能得到确定的结论。
\end{rem}

\begin{proof}
  设 $n$ 是任意正整数,则存在一个双射 $f: \left\{1,2,\dots,n\right\} \rightarrow \left\{n,n-1,\dots,1\right\}$,即将 $\left\{1,2,\dots,n\right\}$ 中的元素按照相反的顺序重新编号,使得 $f(i) = n-i+1$。因此,我们有:
  \begin{align*}
    \sum_{i=1}^n i & = 1 + 2 + \cdots + (n-1) + n           \\
                   & = f(1) + f(2) + \cdots + f(n-1) + f(n) \\
                   & = (n+1) \cdot \frac{n}{2}              \\
                   & = \frac{n(n+1)}{2}
  \end{align*}
  因此,定理成立。
\end{proof}

\clearpage
\section{引用}\label{sec:reference}

在本节(\cref{sec:reference})中,我们将探讨如何在 \LaTeX\ 中进行参考文献引用和交叉引用,以便于读者查阅文献和方便地引用文档中的其他部分。

\subsection{参考文献的引用}

在学术论文或科技报告中,通常需要引用相关文献以支持观点或论证。为了方便读者查阅,我们需要在论文中标注参考文献。在 \LaTeX\ 中,可使用 \verb|\cite| 命令引用参考文献。参考文献的外观应符合国标 GB/T 7714,可使用 \BibLaTeX{} 配合 \pkg{biblatex-gb7714-2015} 样式包来控制输出样式。关于各类文献的标准格式和 \BibLaTeX{} 的使用方法,请参考 \pkg{biblatex-gb7714-2015} 样式包文档的第 4.3 节。

在正文中,可使用 \verb|\cite| 命令引用参考文献。例如:

\begin{Verbatim}
这是一个引用示例 \cite{book1}。
\end{Verbatim}

其中,\verb|book1| 是参考文献的键值,即参考文献文件中每个条目的唯一标识符。引用命令会自动在文中插入相应文献序号,并在文末的参考文献列表中显示相应文献条目。

如果需要同时标注多个参考文献,可使用逗号分隔键值。例如:

\begin{Verbatim}
这是一个引用示例 \cite{book1,online1}。
\end{Verbatim}

请注意,本模板所用的所有文献均为ChatGPT生成的。我们不保证这些文献的真实性。

使用 \verb|\cite{key1,key2,key3...}| 命令可在正文中产生带有括号的上标引用的参考文献,如\cite{book1,online1,article1}。以下是使用 \verb|\cite| 命令的引用示例:
\begin{itemize}
  \item 普通图书\cite{book1,book2}
  \item 论文集、会议录\cite{conf1,conf2}
  \item 科技报告\cite{techreport1,techreport2}
  \item 学位论文\cite{thesis1,thesis2,thesis3}
  \item 专利文献\cite{patent1,patent2}
  \item 专著中析出的文献\cite{inbook1,inbook2}
  \item 期刊中析出的文献\cite{qin2021,article1,article2}
  \item 报纸中析出的文献\cite{newspaper1,newspaper2}
  \item 电子文献\cite{online1,online2,online3}
\end{itemize}

使用 \verb|\parencite{key1,key2,key3...}| 命令可以在正文中产生带有括号的引用参考文献。下面是使用 \verb|\parencite| 命令的引用示例:
\begin{itemize}
  \item 普通图书\parencite{book1,book2}
  \item 论文集、会议录\parencite{conf1,conf2}
  \item 科技报告\parencite{techreport1,techreport2}
  \item 学位论文\parencite{thesis1,thesis2,thesis3}
  \item 专利文献\parencite{patent1,patent2}
  \item 专著中析出的文献\parencite{inbook1,inbook2}
  \item 期刊中析出的文献\parencite{qin2021,article1,article2}
  \item 报纸中析出的文献\parencite{newspaper1,newspaper2}
  \item 电子文献\parencite{online1,online2,online3}
\end{itemize}


可使用 \verb|\nocite{key1,key2,key3...}| 将参考文献条目加入文献表中,但不在正文中引用。使用 \verb|\nocite{*}| 可将参考文献数据库中的所有条目加入文献表中。

当我们想在正文(非参考文献章节)中插入对某一参考文献的完整引用时,可以使用 \verb|\fullcite{key1}| 命令。下面是使用 \verb|\fullcite| 命令的引用示例:

\begin{itemize}
    \item \fullcite{qin2021}
\end{itemize}

有时,我们想要用脚注的形式引用某参考文献,那么我们可以使用 \verb|\footfullcite{key1}| 命令\footfullcite{qin2021}。

另外,为了避免遗漏引用,有时候需要使用文献管理工具来管理参考文献,例如 Zotero 或 EndNote 等。这些工具可以帮助你轻松管理文献数据库,生成参考文献列表,甚至将参考文献直接插入到文档中。

无论使用何种方法,都应该注意文献的准确性和完整性。在引用文献时,应尽可能使用最新的版本,并注意对文献信息的正确引用和格式化。这样可以提高文献的可信度,也能为读者提供更多参考信息。

\subsection{脚注}

脚注是一种在文本底部添加注释或补充说明的方式。在 \LaTeX\ 中,可以使用 \verb|\footnote| 命令添加脚注。

例如,在文本中需要添加脚注时,可以在需要添加脚注的单词或句子后面使用 \verb|\footnote| 命令,如下所示:

\begin{Verbatim}
脚注是一种在文本底部添加注释或补充说明的方式\footnote{通常,我们在脚注里也写完整的句子。在文本中使用脚注时,应该遵循学术规范,尽可能引用可信的来源,并注明出处。}。
\end{Verbatim}

其中,花括号中的文本就是脚注的内容。编译文档后,脚注会出现在页面底部,并自动标上数字,如下所示:

脚注是一种在文本底部添加注释或补充说明的方式\footnote{通常,我们在脚注里也写完整的句子。在文本中使用脚注时,应该遵循学术规范,尽可能引用可信的来源,并注明出处。}。

需要注意的是,在使用脚注时,应该尽量避免使用过多的脚注,以免影响文本的阅读体验。同时,脚注的内容应该尽可能简洁明了,突出重点,有助于读者理解和记忆文本内容。

另外,需要注意的是,在文本中使用脚注时,应该遵循学术规范,尽可能引用可信的来源,并注明出处。同时,对于自己的观点和推断,应该明确标注为个人观点,以免误导读者。

\subsection{交叉引用}

在文档中,交叉引用是指引用文档中的某个标签或标记,例如章节、图表、公式或页码等。在 \LaTeX\ 中,可以使用 \verb|\label| 命令为文档中的对象添加标签,使用 \verb|\ref| 命令或 \verb|\pageref| 命令进行引用。

例如,在文档中的某个章节中添加一个标签:

\begin{Verbatim}
\section{引言}
\label{sec:intro}
这是一段引言。
\end{Verbatim}

然后,在文档的其他位置使用 \verb|\ref| 命令来引用这个标签:

\begin{Verbatim}
请参见第 \ref{sec:intro} 章节。
\end{Verbatim}

这样,就可以在文档中产生“请参见第 X 章节”的效果,其中 X 为标签所在章节的编号。

类似地,使用 \verb|\pageref| 命令可以引用页面编号。例如:

\begin{Verbatim}
请参见第 \pageref{sec:intro} 页。
\end{Verbatim}

这样,就可以在文档中产生“请参见第 X 页”的效果,其中 X 为标签所在页码的编号。

交叉引用在文档中很常用,可以帮助读者快速定位到相关内容,提高文档的可读性。但是,需要注意标签的唯一性和正确性,以及引用命令的正确使用方式。在使用交叉引用时,建议先编译一次文档,然后再编译一次,以确保标签和引用命令的对应关系正确。

当文档中需要交叉引用多个对象时,可以使用 \verb|\cref| 命令来自动根据引用对象的类型生成正确的引用词汇。本模板已经导入了 \pkg{cleveref} 宏包,设定好了一组默认的引用词汇,这些引用词汇可以根据需要进行修改。

在文档中为需要引用的对象添加标签,例如:

\begin{Verbatim}
\section{引言}
\label{sec:intro}
这是一段引言。
\end{Verbatim}

接下来,在文档的其他位置使用 \verb|\cref| 命令引用这个标签:
\begin{Verbatim}
请参见\cref{sec:intro}。
\end{Verbatim}
这样,就可以在文档中产生“请参见章节 X”的效果,其中 X 为标签所在章节的编号。\pkg{cleveref} 宏包会自动根据标签所在对象的类型生成正确的引用词汇。

需要注意的是,如果需要修改引用词汇,可以在导言区添加如下代码:

\begin{Verbatim}
\crefname{对象类型}{引用词汇}{引用词汇复数形式}
\Crefname{对象类型}{引用词汇}{引用词汇复数形式}
\end{Verbatim}

例如,可以使用以下代码将“定理”引用词汇修改为“命题”:
\begin{Verbatim}
\crefname{theorem}{命题}{命题}
\Crefname{theorem}{命题}{命题}
\end{Verbatim}
这样,使用 \verb|\cref| 命令引用定理对象时,就会自动生成“请参见命题 X”的效果。

需要同时引用几个并列的对象时,可以直接使用 \verb|\cref| 命令,例如:
\begin{Verbatim}
请参见\cref{sec:introduction,sec:conclusion}。
请参见\cref{sec:introduction,sec:conclusion}。
请参见\cref{sec:introduction,sec:math,sec:reference,sec:float}。
请参见\cref{sec:introduction,sec:math,sec:reference,,sec:float}。
请参见\cref{sec:introduction,sec:reference,sec:conclusion}。
请参见\cref{sec:introduction,sec:reference,sec:conclusion,%
  algo:algorithm,fig:parallel1,lst:fibonacci}。
\end{Verbatim}

这样,就可以在文档中产生“请参见章节 X 和 Y”的效果,其中 X 和 Y 为标签所在章节的编号:
\begin{itemize}
  \item 请参见\cref{sec:introduction,sec:conclusion}。
  \item 请参见\cref{sec:introduction,sec:conclusion}。
  \item 请参见\cref{sec:introduction,sec:math,sec:reference,sec:float}。
  \item 请参见\cref{sec:introduction,sec:math,sec:reference,,sec:float}。
  \item 请参见\cref{sec:introduction,sec:reference,sec:conclusion}。
  \item 请参见\cref{sec:introduction,sec:reference,sec:conclusion,%
        algo:algorithm,fig:parallel1,lst:fibonacci}。
\end{itemize}

上面的例子中,我们展示了如何使用 \verb|\cref| 命令引用多个并列的对象。在引用多个对象时,\pkg{cleveref} 宏包会自动根据对象类型生成正确的引用词汇,并使用适当的连接词将多个对象连接起来。

总之,交叉引用是撰写文档时非常有用的功能,可以方便读者查找相关内容。在使用交叉引用时,可以选择使用 \verb|\ref| 命令手动指定引用词汇,也可以使用 \verb|\cref| 命令自动根据对象类型生成引用词汇。同时,\pkg{cleveref} 宏包可以帮助用户自动设置正确的引用词汇,提高文档的可读性。
\clearpage
\section{浮动体}\label{sec:float}

在本节(\cref{sec:float})中,我们将介绍浮动体的概念、分类及其在论文中的使用。浮动体包括图片、表格和算法等,它们的共同特点是可以自由浮动在页面中,不会影响正文的排版。

浮动体可以通过插入代码或使用专门的宏包来实现,常用的宏包有 \pkg{graphicx}、\pkg{float} 和 \pkg{algorithm} 等。在插入浮动体时,需要注意设置浮动体的位置和大小,以保证排版效果。

浮动体的正确使用可以提高论文的可读性和美观程度,但也需要注意不要过度使用或使其影响正文的连贯性。

\subsection{插图}

在 \TeX{} 中,插图功能是通过特定编译程序提供的机制实现的。不同的编译程序支持不同的图形格式。通常,图形都放置在浮动环境中。浮动的意思是,最终排版的图形位置不一定与源文件中的位置对应,这也是新手在使用 \LaTeX{} 时可能会遇到的问题。如果需要强制固定浮动图形的位置,请使用 \pkg{float} 宏包,并使用 \texttt{[H]} 参数。

\subsubsection{单个图形}

要插入单个图形,可以使用 \verb|\includegraphics| 命令,并在其参数中指定图像文件名和一些选项。例如:

\begin{figure}[htbp]
  \centering
  \includegraphics[width=0.5\textwidth]{example-image}
  \caption{这是一个示例图形}
  \label{fig:example}
\end{figure}

这段代码会在浮动体中插入一个名为 \verb|example-image| 的示例图形,并将其居中显示。图形下方会有一个标题“这是一个示例图形”,同时为该图形标上标签,以便于交叉引用。

需要注意的是,在使用 \verb|\includegraphics| 命令时,应该将图像文件放在与主文档相同的目录中,或者在参数中指定文件的完整路径。

\subsubsection{多个图形}

如果需要在同一浮动体中插入多个图形,可以使用 \verb|\subfigure| 命令或 \pkg{subcaption} 宏包。这些方法可以按照一定的排列方式组合多个图形,并分别为每个子图形添加标题和标签。

例如,可以使用 \verb|\subfigure| 命令插入两个水平并列的子图,如\cref{fig:example-subfig-1} 所示。这两个子图共用一个图形计数器,没有各自的子图题。

\begin{figure}[!htp]
  \centering
  \includegraphics[width=0.4\textwidth]{example-image-a}
  \hspace{1cm}
  \includegraphics[width=0.4\textwidth]{example-image-b}
  \caption{这是一个示例图形组}
  \label{fig:example-subfig-1}
\end{figure}

如果多个图形相互独立,不共用一个图形计数器,可以使用 \texttt{minipage} 或 \texttt{parbox},如\cref{fig:parallel1} 和\cref{fig:parallel2}。

\begin{figure}[!htp]
  \begin{minipage}{0.4\textwidth}
    \centering
    \includegraphics[width=\textwidth]{example-image-a}
    \caption{并排第一个图}
    \label{fig:parallel1}
  \end{minipage}\hfill
  \begin{minipage}{0.4\textwidth}
    \centering
    \includegraphics[width=\textwidth]{example-image-b}
    \caption{并排第二个图}
    \label{fig:parallel2}
  \end{minipage}
\end{figure}

为共用一个计数器的多个子图添加子图题,建议使用较新的 \pkg{subcaption} 宏包,不建议使用 \pkg{subfigure} 或 \pkg{subfig} 等宏包。

\pkg{subcaption} 宏包提供了 \cs{subcaptionbox} 命令,可以方便地实现子图的并排排列,并为每个子图添加子图题,如\cref{fig:subcaptionbox} 所示。此外,也可以使用 \pkg{subcaption} 宏包的 \cs{subcaption} 命令,将子图题放在子图之下,子图号用 a)、b) 等表示。可以将多个子图放在 \texttt{minipage} 中,用法与 \cs{caption} 相同。

\begin{figure}[!htp]
  \centering
  \subcaptionbox{左图}{\includegraphics[width=0.4\textwidth]{example-image-a}}
  \hspace{1cm}
  \subcaptionbox{右图}{\includegraphics[width=0.3\textwidth]{example-image-b}}
  \caption{包含子图题的示例(使用 subcaptionbox)}
  \label{fig:subcaptionbox}
\end{figure}

另外,\pkg{subcaption} 宏包还提供了 \pkg{subfigure} 和 \pkg{subtable} 环境,如\cref{fig:subfigure} 所示。这些环境的使用方法与 \pkg{subcaptionbox} 相似,可以实现子图的并排排列,并为每个子图添加子图题。

\begin{figure}[!htp]
  \centering
  \begin{subfigure}{0.3\textwidth}
    \centering
    \includegraphics[width=\textwidth]{example-image-a}
    \caption{左图}
  \end{subfigure}
  \hspace{1cm}
  \begin{subfigure}{0.4\textwidth}
    \centering
    \includegraphics[width=\textwidth]{example-image-b}
    \caption{右图}
  \end{subfigure}
  \caption{包含子图题的范例(使用 subfigure)}
  \label{fig:subfigure}
\end{figure}



总之,在使用 \LaTeX{} 插图时,应该注意浮动体的特性,尽量使用 \pkg{float} 宏包进行浮动体的控制,避免出现排版问题。同时,也应该熟悉插图命令的使用方法,以便于插入单个或多个图形,并设置适当的标题和标签。


\subsection{表格}

\subsubsection{基本表格}

编排表格时应简单明了、表达一致,内容明晰易懂,表文呼应,内容一致。表题应置于表格上方。

表格的编排建议采用国际通行的三线表\footnote{三线表以其形式简洁、功能分明、阅读方便而在科技论文中被推荐使用。三线表通常只有3条线,即顶线、底线和栏目线,没有竖线。}。可以使用 \pkg{booktabs} 宏包提供的 \cs{toprule}、\cs{midrule} 和 \cs{bottomrule} 命令来绘制三线表。同时,\pkg{longtable} 宏包可以与三线表很好地配合使用,可以排版较长的表格。

要编写三线表,需要在表格的头部、底部和每一栏之间绘制水平线。其中,\cs{toprule} 绘制顶部线,\cs{midrule} 绘制中部线,\cs{bottomrule} 绘制底部线。可以按照如下方式绘制三线表:
\begin{table}[!hpt]
  \caption{一个三线表}
  \label{tab:firstone}
  \centering
  \begin{tabular}{@{}llr@{}} \toprule
    \multicolumn{2}{c}{\textbf{Item}}                            \\ \cmidrule(r){1-2}
    \textbf{Animal} & \textbf{Description} & \textbf{Price} (\$) \\ \midrule
    Gnat            & per gram             & 13.65               \\
                    & each                 & 0.01                \\
    Gnu             & stuffed              & 92.50               \\
    Emu             & stuffed              & 33.33               \\
    Armadillo       & frozen               & 8.99                \\ \bottomrule
  \end{tabular}
\end{table}

\subsubsection{复杂表格}

我们经常会在表格下方标注数据来源,或者对表格里面的条目进行解释。可以用
\pkg{threeparttable} 实现带有脚注的表格,如\cref{tab:bigtable}。

\begin{table}[!htbp]
  \centering
  \begin{threeparttable}[b]
    \caption{一个更大的三线表}
    \label{tab:bigtable}
    \begin{tabular}{@{}llrrrrrrr@{}} \toprule
      \multicolumn{2}{c}{\textbf{Item}} & \multicolumn{2}{c}{\textbf{Category 1}} & \multicolumn{2}{c}{\textbf{Category 2}} & \multicolumn{2}{c}{\textbf{Category 3}} & \multicolumn{1}{c}{\textbf{Total}}                                                                                     \\ \cmidrule(r){1-2} \cmidrule(lr){3-4} \cmidrule(lr){5-6} \cmidrule(lr){7-8} \cmidrule(l){9-9}
      \textbf{Animal}                   & \textbf{Description}                    & \textbf{Price} (\$)                     & \textbf{Quantity}                       & \textbf{Price} (\$)                & \textbf{Quantity} & \textbf{Price} (\$) & \textbf{Quantity} & \textbf{Price} (\$) \\ \midrule
      Gnat                              & per gram\tnote{a}                       & 13.65                                   & 100                                     & 12.35                              & 200               & 11.55               & 150               & 3650.00             \\
                                        & each                                    & 0.01                                    & 5000                                    & 0.01                               & 10000             & 0.009               & 20000             & 550.00              \\
      Gnu                               & stuffed                                 & 92.50                                   & 10                                      & 94.50                              & 15                & 96.50               & 20                & 5815.00             \\
      Emu                               & stuffed                                 & 33.33                                   & 25                                      & 34.33                              & 30                & 35.33               & 35                & 2704.95             \\
      Armadillo                         & frozen                                  & 8.99                                    & 50                                      & 7.99                               & 40                & 6.99                & 30\tnote{b}       & 1094.50             \\ \bottomrule
    \end{tabular}
    \begin{tablenotes}
      \item [a] 第一条脚注。
      \item [b] 第二条脚注。
    \end{tablenotes}
  \end{threeparttable}
\end{table}

为了编写更复杂的表格,我们可以使用 \href{https://www.tablesgenerator.com/}{Tables Generator} 来生成 \LaTeX{} 代码。该工具可以方便地设置表格的列数、行数、内容、格式、颜色等属性,支持常用的表格类型和样式,并且可以实时生成 \LaTeX{} 代码和预览效果,避免手动编写 \LaTeX{} 代码时出错。


如某个表需要转页接排,可以用 \pkg{longtable} 实现。接排时表题省略,表头应重复书
写,并在右上方写“续表 X”,如\cref{tab:performance}。

\begin{ThreePartTable}
  \begin{TableNotes}
    \item[a] 一个脚注
    \item[b] 另一个脚注
  \end{TableNotes}
  \begin{longtable}[c]{c*{6}{r}}
    \caption{实验数据}
    \label{tab:performance}                                                                                                                      \\
    \toprule
    \textbf{测试程序\tnote{b}} & \multicolumn{1}{c}{\textbf{正常运行}}   & \multicolumn{1}{c}{\textbf{同步}}
                           & \multicolumn{1}{c}{\textbf{检查点}}    & \multicolumn{1}{c}{\textbf{卷回恢复}}
                           & \multicolumn{1}{c}{\textbf{进程迁移}}   & \multicolumn{1}{c}{\textbf{检查点}}                                              \\
                           & \multicolumn{1}{c}{\textbf{时间} (s)} & \multicolumn{1}{c}{\textbf{时间} (s)}
                           & \multicolumn{1}{c}{\textbf{时间} (s)} & \multicolumn{1}{c}{\textbf{时间} (s)}
                           & \multicolumn{1}{c}{\textbf{时间} (s)} & \textbf{文件}(KB)                                                               \\
    \midrule
    \endfirsthead
    \multicolumn{7}{r}{\textbf{续表~\thetable}}                                                                                                    \\

    \toprule
    \textbf{测试程序\tnote{b}} & \multicolumn{1}{c}{\textbf{正常运行}}   & \multicolumn{1}{c}{\textbf{同步}}
                           & \multicolumn{1}{c}{\textbf{检查点}}    & \multicolumn{1}{c}{\textbf{卷回恢复}}
                           & \multicolumn{1}{c}{\textbf{进程迁移}}   & \multicolumn{1}{c}{\textbf{检查点}}                                              \\
                           & \multicolumn{1}{c}{\textbf{时间} (s)} & \multicolumn{1}{c}{\textbf{时间} (s)}
                           & \multicolumn{1}{c}{\textbf{时间} (s)} & \multicolumn{1}{c}{\textbf{时间} (s)}
                           & \multicolumn{1}{c}{\textbf{时间} (s)} & \textbf{文件}(KB)                                                               \\
    \midrule
    \endhead
    \hline
    \multicolumn{7}{r}{续下页}
    \endfoot
    \insertTableNotes
    \endlastfoot
    CG.C.2                 & 23.05                               & 0.002                               & 0.116          & 0.035 & 0.589 & 32491  \\
    CG.A.4                 & 15.06                               & 0.003                               & 0.067          & 0.021 & 0.351 & 18211  \\
    CG.A.8                 & 13.38                               & 0.004                               & 0.072          & 0.023 & 0.210 & 9890   \\
    CG.B.2                 & 867.45                              & 0.002                               & 0.864          & 0.232 & 3.256 & 228562 \\
    CG.B.4                 & 501.61                              & 0.003                               & 0.438          & 0.136 & 2.075 & 123862 \\
    CG.B.8                 & 384.65                              & 0.004                               & 0.457          & 0.108 & 1.235 & 63777  \\
    MG.A.2                 & 112.27                              & 0.002                               & 0.846\tnote{a} & 0.237 & 3.930 & 236473 \\
    MG.A.4                 & 59.84                               & 0.003                               & 0.442          & 0.128 & 2.070 & 123875 \\
    MG.A.8                 & 31.38                               & 0.003                               & 0.476          & 0.114 & 1.041 & 60627  \\
    MG.B.2                 & 526.28                              & 0.002                               & 0.821          & 0.238 & 4.176 & 236635 \\
    MG.B.4                 & 280.11                              & 0.003                               & 0.432          & 0.130 & 1.706 & 123793 \\
    MG.B.8                 & 148.29                              & 0.003                               & 0.442          & 0.116 & 0.893 & 60600  \\
    LU.A.2                 & 2116.54                             & 0.002                               & 0.110          & 0.030 & 0.532 & 28754  \\
    LU.A.4                 & 1102.50                             & 0.002                               & 0.069          & 0.017 & 0.255 & 14915  \\
    LU.A.8                 & 574.47                              & 0.003                               & 0.067          & 0.016 & 0.192 & 8655   \\
    LU.B.2                 & 9712.87                             & 0.002                               & 0.357          & 0.104 & 1.734 & 101975 \\
    LU.B.4                 & 4757.80                             & 0.003                               & 0.190          & 0.056 & 0.808 & 53522  \\
    LU.B.8                 & 2444.05                             & 0.004                               & 0.222          & 0.057 & 0.548 & 30134  \\
    EP.A.2                 & 123.81                              & 0.002                               & 0.010          & 0.003 & 0.074 & 1834   \\
    EP.A.4                 & 61.92                               & 0.003                               & 0.011          & 0.004 & 0.073 & 1743   \\
    EP.A.8                 & 31.06                               & 0.004                               & 0.017          & 0.005 & 0.073 & 1661   \\
    EP.B.2                 & 495.49                              & 0.001                               & 0.009          & 0.003 & 0.196 & 2011   \\
    SP.B.4                 & 397.69                              & 0.002                               & 0.015          & 0.005 & 0.122 & 1763   \\
    SP.B.8                 & 196.74                              & 0.003                               & 0.018          & 0.006 & 0.082 & 1865   \\
    AA.A.2                 & 13.81                               & 0.002                               & 0.012          & 0.004 & 0.074 & 1362   \\
    AA.A.4                 & 6.92                                & 0.003                               & 0.011          & 0.003 & 0.073 & 1331   \\
    AA.A.8                 & 3.06                                & 0.004                               & 0.017          & 0.004 & 0.073 & 1225   \\
    AA.B.2                 & 49.49                               & 0.001                               & 0.009          & 0.003 & 0.196 & 2254   \\
    AA.B.4                 & 24.69                               & 0.002                               & 0.012          & 0.004 & 0.122 & 1453   \\
    AA.B.8                 & 12.74                               & 0.003                               & 0.018          & 0.005 & 0.082 & 1432   \\
    MG.A.2                 & 112.27                              & 0.002                               & 0.846          & 0.237 & 3.930 & 236473 \\
    MG.A.4                 & 59.84                               & 0.003                               & 0.442          & 0.128 & 2.070 & 123875 \\
    MG.A.8                 & 31.38                               & 0.003                               & 0.476          & 0.114 & 1.041 & 60627  \\
    MG.B.2                 & 526.28                              & 0.002                               & 0.821          & 0.238 & 4.176 & 236635 \\
    MG.B.4                 & 280.11                              & 0.003                               & 0.432          & 0.130 & 1.706 & 123793 \\
    MG.B.8                 & 148.29                              & 0.003                               & 0.442          & 0.116 & 0.893 & 60600  \\
    LU.A.2                 & 2116.54                             & 0.002                               & 0.110          & 0.030 & 0.532 & 28754  \\
    LU.A.4                 & 1102.50                             & 0.002                               & 0.069          & 0.017 & 0.255 & 14915  \\
    LU.A.8                 & 574.47                              & 0.003                               & 0.067          & 0.016 & 0.192 & 8655   \\
    LU.B.2                 & 9712.87                             & 0.002                               & 0.357          & 0.104 & 1.734 & 101975 \\
    LU.B.4                 & 4757.80                             & 0.003                               & 0.190          & 0.056 & 0.808 & 53522  \\
    LU.B.8                 & 2444.05                             & 0.004                               & 0.222          & 0.057 & 0.548 & 30134  \\
    EP.A.2                 & 123.81                              & 0.002                               & 0.010          & 0.003 & 0.074 & 1834   \\
    \bottomrule
  \end{longtable}
\end{ThreePartTable}



\subsection{算法环境}

算法环境可以使用 \pkg{algorithms} 宏包或者较新的 \pkg{algorithm2e} 实现。
\cref{algo:algorithm} 是一个使用 \pkg{algorithms} 的例子。关于排版算法环境
的具体方法,请阅读相关宏包的官方文档。

\begin{algorithm}
  \caption{计算斐波那契数列}\label{algo:algorithm}
  \begin{algorithmic}[1]
    \Require $n \geq 0$
    \Ensure $fib(n)$
    \Function{Fibonacci}{$n$}
    \If{$n \leq 1$}
    \State \Return $n$
    \Else
    \State \Return \Call{Fibonacci}{$n-1$} + \Call{Fibonacci}{$n-2$}
    \EndIf
    \EndFunction
  \end{algorithmic}
\end{algorithm}

\subsection{代码环境}

虽然我们可以在论文中加入算法,但是并不推荐添加过多的代码。如果确实需要在论文中添加代码,建议采用 \pkg{minted} 宏包,该宏包可以提供代码高亮的功能,使得代码更加易于阅读。

为了让代码可以在论文中正确地排版,我们使用了 \pkg{listings} 宏包来包装 \pkg{minted}。具体来说,我们在 \texttt{listing} 环境中嵌套了 \texttt{minted} 环境来展示代码,并使用 \texttt{caption} 命令来添加代码标题。代码的标签和交叉引用可以使用 \texttt{label} 和 \texttt{cref} 命令来实现。

\cref{algo:algorithm} 的Python代码如\cref{lst:fibonacci} 所示。



\begin{listing}[htbp]
  \begin{minted}{python}
def fibonacci(n: int) -> int:
    """
    Calculates the nth Fibonacci number.

    Args:
        n (int): Index of the desired Fibonacci number. Must be non-negative.

    Returns:
        int: The nth Fibonacci number.

    Raises:
        ValueError: If the input is negative.
    """
    if n < 0:
        raise ValueError("Negative arguments are not supported")
    if n == 0:
        return 0
    if n == 1:
        return 1
    return fibonacci(n - 1) + fibonacci(n - 2)
  \end{minted}
  \caption{计算斐波那契数列的Python实现}\label{lst:fibonacci}
\end{listing}

\subsection{\texttt{\textbackslash{}clearpage} 与 \texttt{\textbackslash{}newpage}}

在 \LaTeX\ 中,使用 \verb|\newpage| 命令可以在当前位置开始新的一页。而使用 \verb|\clearpage| 命令不仅可以开始新的一页,还可以将未处理的浮动体(例如图表等)全部输出,确保当前页面上的所有浮动体都已经处理完毕。

因此,在需要控制页面排版时,应该优先使用 \verb|\clearpage| 命令,而不是简单地使用 \verb|\newpage| 命令。如果不使用 \verb|\clearpage| 命令,可能会导致未处理的浮动体出现在不合适的位置,影响文档的排版效果。

另外,需要注意的是,如果在文档中使用了浮动体(例如图表等),那么最好将浮动体放在一个浮动体环境中,并设置适当的位置选项(例如 \verb|[htbp]| 等),以控制浮动体的排版位置。在需要开始新的一页时,应该尽可能使用 \verb|\clearpage| 命令,以确保当前页面上的所有浮动体都已经处理完毕,从而避免出现排版问题。

综上所述,使用 \verb|\clearpage| 命令可以确保当前页面上的所有浮动体都已经处理完毕,从而避免出现排版问题。在文档中使用浮动体时,应该将浮动体放在一个浮动体环境中,并设置适当的位置选项,以控制浮动体的排版位置。
\clearpage
\input{sections/05_conclusion}
\clearpage

\phantomsection % 添加占位符章节
\addcontentsline{toc}{section}{参考文献}
\printbibliography[heading=bibliography,title=参考文献]

\clearpage
\input{sections/thanks}

\end{document}
