\section{毕业设计(论文)课题背景(含文献综述)}

\subsection{研究背景和意义}

在这一部分,您需要详细阐述选择此课题进行研究的背景和动机。说明这一课题在现实社会、科学技术发展或学术领域中的重要性和必要性。例如,您可以从以下几个方面来展开:
\begin{itemize}
  \item 社会或科技发展中存在的问题和需求;
  \item 相关领域的研究空白;
  \item 该课题可能带来的创新点和实际应用价值。
\end{itemize}

建议详细描述课题的研究背景,使读者能够清晰地了解您选择此课题的原因和动机。

\subsection{国内外研究现状和文献综述}
这一部分要求您对国内外在您的研究领域内的研究现状进行全面的回顾和总结,主要包括:
\begin{itemize}
  \item 描述相关领域的研究进展,特别是近几年的发展趋势;
  \item 分析和比较不同研究者的研究成果,指出其优势和不足;
  \item 引用关键文献,总结现有研究的共识和争议;
  \item 明确指出您的研究将如何在现有基础上进行创新或填补研究空白。
\end{itemize}

此部分要求深入细致,不仅要有广度,还要有深度。建议您在撰写此部分时,要充分利用图表,以便更形象、更直观地展示研究现状和发展趋势。
